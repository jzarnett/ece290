\input{configuration}

\title{Lecture 6 --- Statue of Frauds, Parol Evidence Rule }

\author{Jeff Zarnett \\ \small \texttt{jzarnett@uwaterloo.ca}}
\institute{Department of Electrical and Computer Engineering \\
  University of Waterloo}
\date{\today}


\begin{document}

\begin{frame}
  \titlepage

\begin{center}
  \small{Acknowledgments: Douglas Harder~\cite{dwh}, Julie Vale~\cite{jv}}
  \end{center}
\end{frame}

\part{Statute of Frauds}

\begin{frame}
\partpage
\end{frame}



\begin{frame}
\frametitle{Contracts: Oral \& Written}

As we have discussed, contracts may be written or verbal.

Thus, they can be entered into either entirely through discussions or through correspondence (written communication).

This is common law precedent.

The legislature can pass laws that formalize or supersede precedent.

\end{frame}



\begin{frame}
\frametitle{Substance and Form}

A contract's \alert{substance}, the terms of the contract, may exist independent of any physical \alert{form}.

Contracts can be~\cite{lba}:

\begin{enumerate}
	\item Entirely oral
	\item Partly oral, partly written
	\item Entirely in writing, in one document
	\item Entirely in writing, spread across multiple documents.
\end{enumerate}

All are potentially valid forms of a contract.

\end{frame}

\begin{frame}
\frametitle{Written Form}

It's usually a good idea, of course, to write the contract down.

The more complex it is, the more important it is to write it down.

Human memories are fallible.\\
\quad And contracts may last over a period of years.


\end{frame}



\begin{frame}
\frametitle{I'm a Medieval Man}

In 1677 the English parliament introduced an act called the \alert{Statute of Frauds}.

The act came from times of turmoil and dealt with, mainly, ownership and transfer of ownership of land.

Parliament wished to settle the issue of who owned real estate and eliminate \alert{perjured} (false) testimony through an act.

The original act was not brilliantly written, but there is an Ontario Statute of Frauds, which is quite short; no more than two pages.

\end{frame}



\begin{frame}
\frametitle{The Statute of Frauds}

The full text of the statute can be read at~\cite{sof}.

In short: the statute requires certain contracts to be in writing.

A contract that falls under the statute is not enforceable unless it is in writing.

\end{frame}



\begin{frame}
\frametitle{Statute of Frauds, Section 4}


\begin{quote}
No action shall be brought to charge any executor or administrator upon any special promise to answer damages out of the executor's or administrator's own estate, or to charge any person upon any special promise \textbf{to answer for the debt, default or miscarriage of any other person, or to charge any person upon any contract or sale of lands, tenements or hereditaments}, or any interest in or concerning them, unless the agreement upon which the action is brought, or some memorandum or note thereof is in writing and signed by the party to be charged therewith or some person thereunto lawfully authorized by the party.~\cite{sof}
\end{quote}

\end{frame}



\begin{frame}
\frametitle{The Struggle is Real}

Some argue that the Statute of Frauds actually encourages fraud~\cite{lba}.

Parties to oral contracts have sometimes been able to avoid their obligations because it was included in the statute.

In this way, the courts have tried to limit the application of the Statue.


The statute will only apply, of course, if the contract would otherwise be valid.\\
\quad There is no contact to be considered unless a valid contract has been formed.

\end{frame}



\begin{frame}
\frametitle{Real Property}

The statute, as we know, is primarily concerned with land (real property).

In any matters concerning real property, you need legal advice.

This course cannot provide adequate guidance on this complicated subject, so it is best not to go into it.

\end{frame}



\begin{frame}
\frametitle{Taking on Others' Debt}

There are two mechanisms for taking on someone else's debt.

1. \alert{Guarantee}: a promise to pay if the debtor defaults.\\
\quad Must be in writing.

Example: you want to rent an apartment and the landlord requires a parent to guarantee payment of rent.

2. \alert{Indemnity}: taking on primary responsibility for the debts of another.\\
\quad Need not be in writing.

A promise by the purchaser of a business to pay back wages of the employees would be a promise to indemnify~\cite{lba}.

\end{frame}



\begin{frame}
\frametitle{Void vs Unenforceable}

Remember that the statute says a valid contract concerning, say, transfer of land must be in writing to be enforceable.

A contract may still exist, even if it is not enforceable.

If John and Jane enter into a valid verbal contract to transfer land and both do their part, is the contract legal?

What if John, partway through, decides to back out of the deal?

\end{frame}



\begin{frame}
\frametitle{Void vs Unenforceable}

The contract, if completed, is legal. Remember, the contract is valid. The courts will not enforce it if John and Jane did not put it in writing.

Let's say nothing has happened yet but John wants to back out. 

That is not very nice, but the courts will not act to enforce the deal.

What if, however, Jane has paid John a deposit and then John wants to back out of the deal?

Can he keep the deposit and laugh in Jane's face?

\end{frame}



\begin{frame}
\frametitle{Void vs Unenforceable}

He could repay the deposit (it sounds like the right thing to do, no)?\\
\quad The courts won't make him go through with the deal.

But he cannot keep the deposit. The contract is unenforceable, but not void. 

The courts will allow Jane to sue for and recover the deposit she paid.

A contract existed and while it will not be enforced, it can be taken as evidence in Jane's lawsuit.

(Also, John is probably in trouble criminally, but that's beyond this course.)

\end{frame}



\begin{frame}
\frametitle{Life Pro Tips}

It is highly recommended that all contracts be in writing.\\
\quad Even if that means asking a lawyer to draft or review it.

Lawyers charge hundreds of dollars an hour. It is much cheaper to get a contract in writing than to litigate the issue later.

An ounce (28g) of prevention is worth a pound (454g) of cure!

\end{frame}


\part{Parol Evidence Rule}

\begin{frame}
\partpage
\end{frame}



\begin{frame}
\frametitle{Opening Scenario}

Imagine you want to rent an apartment. You negotiate with the landlord about price and in your discussions, the landlord promises to replace the old refrigerator.

Then in the written contract you sign, the term about replacement of the refrigerator does not appear.

Can you sue the landlord to have the refrigerator replaced?

Does it matter if it is in writing or not?

\end{frame}



\begin{frame}
\frametitle{``Completeness'' of Contract}

No, on both counts.

When a contract is established, it is the basis for any future interpretations of what the parties were supposed to do.

If a term is included it is assumed agreed by all parties.

If a term is omitted, it is assumed that the parties agreed to such omissions.

The contract is considered ``complete'' as it is.

\end{frame}



\begin{frame}
\frametitle{Other Evidence}

Any discussions, communications, letters of intent, etc. prepared prior to the contract, in general, cannot be used in any way to interpret the contract.

The doctrine that the contract forms a common basis for the agreement is known as the \alert{parol evidence rule}.

This doctrine prevents any party to a written contract from presenting past evidence that contradicts or adds to the terms of a contract.

\end{frame}



\begin{frame}
\frametitle{Exceptions}

There are a number of exceptions to this rule that we will discuss.

Keep in mind they are all ``special cases'' and the courts will generally enforce just what was written.

The courts are reluctant to relax this rule because they fear this would tempt parties to make up stories about favourable terms~\cite{lba}.

It is better to break off negotiations than enter into a contract without an important term written down.

\end{frame}



\begin{frame}
\frametitle{Alex and Bill}

Let's look at an example from~\cite{lba}.

Suppose that Alex owns a fleet of dump trucks comes to an agreement orally with Bill, a paving contractor, to move 25~000 cubic feet of gravel for \$15~000.

To finance paving operations Bill needs a bank loan, and the bank requests evidence that this will go ahead.

Alex sends a note to Bill saying that he will deliver 5~000 cubic metres for \$3~000 in the next month.

Alex discovered the price he quoted was too low. 

Alex decides to claim that the agreement with Bill has been reduced to writing and he need only live up to the 5~000 cubic metres contract.

Why does the rule not apply?

\end{frame}



\begin{frame}
\frametitle{Scope of the Rule}

Key idea: a contract can be partly oral and partly in writing.

The parol evidence rule is only applicable here if the written document is the whole contract.

The document in question was drawn up so Alex and Bill could complete their oral contract, not to represent the whole contract.

\end{frame}



\begin{frame}
\frametitle{Resolution of Ambiguities}

Oral evidence can be used not to change the term of the contract, but to interpret what the parties meant. 

Suppose a contract related to kitchen cabinets contains the term ``build''.

What is the exact meaning of this word? Does it include assembly? 

\end{frame}



\begin{frame}
\frametitle{\textit{Pym v. Campbell}}

Consider the 1865 case of \textit{Pym v. Campbell}.

Pym wished to sell a stake in an invention to Campbell. 

Campbell wanted two engineers to examine this invention and approve of it. 

This was before the age of the cell phone so it was not the case that both engineers were present and available at the time.

The first engineer expressed a favourable opinion.

In the interests of timeliness the parties signed the contract with the oral stipulation that the engineer would need to approve.

Eventually, the second engineer examined the invention and did not approve.

\end{frame}



\begin{frame}
\frametitle{\textit{Pym v. Campbell}}

Given that the second engineer did not approve, Campbell did not wish to purchase it.

Pym sued and entered into evidence the signed contract.

The parol evidence rule says the oral evidence that it was conditional upon the second engineer's approval did not appear in the contract and cannot count.

Is this the correct interpretation? Why or why not?

\end{frame}



\begin{frame}
\frametitle{\textit{Pym v. Campbell}}

No - the rule does not apply here. 

In this case the oral evidence is not about the terms of the contract.

Instead, it indicates that Pym and Campbell did not have mutual intent to enter into a contract.

That being absent, no valid contract was formed!

\end{frame}



\begin{frame}
\frametitle{Collateral Agreements}

The parol evidence rule precludes modification of an existing contract based on evidence from before the contract is signed.

It does not mean that a separate, parallel agreement may not be formed. 

This separate agreement, even if oral, could be enforceable...\\
\quad As long as it is a valid contract on its own merits.

\end{frame}


\begin{frame}
\frametitle{Ex Post Facto}

One final thing to note: the parol evidence rule refers to things before the formation of the original contract.

A later agreement may modify, change, or rescind the original agreement.

The parties may agree, after the signing, to add or modify terms (in writing or orally) and evidence of that will be valid if the contract is later disputed.


\end{frame}


\begin{frame}
\frametitle{References \& Disclaimer}
\bibliographystyle{ieeetr}
\setbeamertemplate{bibliography item}{\insertbiblabel}
{\scriptsize
\bibliography{290}
}
\vfill

{\tiny Disclaimer: the material presented in these lectures slides is intended for use in the course ECE~290 at the University of Waterloo and should not be relied upon as legal advice. Any reliance on these course slides by any party for any other purpose are the responsibility of such parties.  The author(s) accept(s) no responsibility for damages, if any, suffered by any party as a result of decisions made or actions based on these course slides for any other purpose than that for which it was intended.\par}


\end{frame}


\end{document}

