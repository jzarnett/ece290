\input{configuration}

\title{Lecture 18 --- Concurrent Tortfeasors and Contributory Negligence }

\author{Jeff Zarnett \\ \small \texttt{jzarnett@uwaterloo.ca}}
\institute{Department of Electrical and Computer Engineering \\
  University of Waterloo}
\date{\today}


\begin{document}

\begin{frame}
  \titlepage

\begin{center}
  \small{Acknowledgments: Douglas Harder~\cite{dwh}, Julie Vale~\cite{jv}}
  \end{center}
\end{frame}



\begin{frame}
\frametitle{You Get Fault! Everyone Gets Fault!}

In some of the previous examples we said that liability was spread across several parties in various proportions.

There are rules on how fault is apportioned.

The official term for this is \alert{Concurrent Tortfeasors}.

\end{frame}



\begin{frame}
\frametitle{Concurrent Tortfeasors}

\textit{Corporation of District of Surry v. Carrol-Hatch et al.}, 1979

In designing a Police Station, the architect and engineers made two shallow test pits.

The engineers recommended deep soil tests.

This was rejected by the architect and the engineers submitted their report based on an examination of the shallow pits.

Once the building was complete, settlement require significant additional structural changes.

Deep soil tests would have revealed the issues affecting the building.

\end{frame}




\begin{frame}
\frametitle{Concurrent Tortfeasors}

We expect that the owner was relying on the professional judgement of the architect and engineer.

This resulted in a duty of care.

What percentage of fault will be assigned to the architect \& engineer, and why?

(This topic is concurrent tortfeasors so what are the chances it's 100\%-0\%...)

\end{frame}



\begin{frame}
\frametitle{Concurrent Tortfeasors}

The architect was assigned 60\%; the engineer 40\%.

The engineers could not remove that duty of care by accepting the architect's decision when they did know or should have known better.

They should have insisted on the full, deep soil tests.

\end{frame}



\begin{frame}
\frametitle{Concurrent Tortfeasors}

In 1968's \textit{Long v. Thiessen}, the BC Court of Appeal established a formula~\cite{ct}.

When two tortfeasors are independently liable for separate incidents, the burden is placed on the second tortfeasor to prove damages caused by the first.

This would reduce the burden of the second accordingly.

Does this seem strange? It pits the defendants against one another...

But this was later revised in two subsequent cases.

\end{frame}



\begin{frame}
\frametitle{Concurrent Tortfeasors}

Those two are \textit{Athey v. Leonati} (1996) and \textit{G. (E.D.) v. Hammer} (2003)~\cite{ct}.

Tortfeasors liable for separate acts, which lead to a single injury, are jointly and severally liable to the plaintiff for all of the damages.

(Severally means individually in this sentence.)

Apportionment is then done under the rules of the (provincial) Negligence Act.

\end{frame}



\begin{frame}
\frametitle{Divisibility}

There is often a question of \alert{divisibility}: can the injury suffered by the plaintiff be separated out?

If it is divisible, then each tortfeasor is responsible only for the injury it caused.

If both defendants have contributed to an injury, it is indivisible~\cite{ct}.

In that case, the plaintiff can recover damages from either or both defendants.

\end{frame}



\begin{frame}
\frametitle{Concurrent vs. Consecutive}

Concurrent tortfeasors require that the torts concur (``run together'').

Example from the case \textit{Aberdeen v. Langley (Township)} in 2007~\cite{ct}.

The concurrent tortfeasors were the negligent driver who struck the plaintiff, and the municipality that installed the guard rail.

The negligent acts are entirely different, and yet, they culminate at a single point in time to produce the injury.

\end{frame}



\begin{frame}
\frametitle{Concurrent vs. Consecutive}

If there are consecutive tortfeasors, the torts are not connected via one event.

Consider the 2007 case \textit{Hutchings v. Dow}.

The plaintiff, Hutchings, was injured in a motor vehicle accident in 2001.

Assessment of damages was complicated by the fact that he suffered another injury from an assault in 2003.

It is the fact that the injuries were cumulative that brings these consecutive tortfeasors together, allowing the plaintiff to pursue them in a single action.

\end{frame}



\begin{frame}
\frametitle{Apportionment of Fault}

The judge will assess the degree to which each party is at fault.

Between the concurrent tortfeasors, the test is ``moral blameworthiness'' rather than the degree of contribution to the damages~\cite{ct}.

This usually turns on facts about:

\begin{itemize}
	\item The duty of care owed by the tortfeasor.
	\item The timing (the first tort is usually the most important).
	\item The nature of the conduct (inattention, deliberate, etc.).
\end{itemize}

\end{frame}



\begin{frame}
\frametitle{Your Own Fault}

It has perhaps occurred to you that the plaintiff may also be partially at fault.

This is like concurrent tortfeasors, but is called \alert{contributory negligence}.

Sometimes the injury is, sadly, the plaintiff's own fault.

\end{frame}



\begin{frame}
\frametitle{Contributory Negligence}

Imagine someone carelessly steps out from behind a parked car and the defendant could not have avoided striking him, even if driving slowly~\cite{lba}.

In this case, the action of the plaintiff is injured, yes, but his own negligence in failing to look before crossing the street absolves the defendant of fault.

In the old common law system, if the defendant could demonstrate any liability on the part of the plaintiff, that would lead to dismissal of the case.

The next evolution of this concept was the ``last chance'' theory.\\
\quad But it requires a chain of events to be established.

\end{frame}



\begin{frame}
\frametitle{Contributory Negligence}

The solution was that the provinces introduced laws on apportionment of fault.

Now if the plaintiff is considered to be x\% liable, then he or she can recover only (100-x)\% from the defendant or defendants.

Apportionment of fault in tort is a tremendously complicated subject and it is best to rely on legal advice. 

\end{frame}


\begin{frame}
\frametitle{References \& Disclaimer}
\bibliographystyle{ieeetr}
\setbeamertemplate{bibliography item}{\insertbiblabel}
{\scriptsize
\bibliography{290}
}
\vfill

{\tiny Disclaimer: the material presented in these lectures slides is intended for use in the course ECE~290 at the University of Waterloo and should not be relied upon as legal advice. Any reliance on these course slides by any party for any other purpose are the responsibility of such parties.  The author(s) accept(s) no responsibility for damages, if any, suffered by any party as a result of decisions made or actions based on these course slides for any other purpose than that for which it was intended.\par}


\end{frame}


\end{document}

