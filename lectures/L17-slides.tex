\input{configuration}

\title{Lecture 17 --- Vicarious Liability }

\author{Jeff Zarnett \\ \small \texttt{jzarnett@uwaterloo.ca}}
\institute{Department of Electrical and Computer Engineering \\
  University of Waterloo}
\date{\today}


\begin{document}

\begin{frame}
  \titlepage

\begin{center}
  \small{Acknowledgments: Douglas Harder~\cite{dwh}, Julie Vale~\cite{jv}}
  \end{center}
\end{frame}



\begin{frame}
\frametitle{Vicarious Liability}

If you ask dictionary.com, vicarious comes from the Latin, meaning something performed, exercised, received, or suffered in place of another.

Vicarious liability comes into play in employment situations.

If an employee commits a tort in the course of employment, the employer, through the principle of vicarious liability, is also liable for damage in tort~\cite{lpe}.

Is it fair for the employer to be responsible for the employee's wrongdoing?

\end{frame}



\begin{frame}
\frametitle{Vicarious Liability}

There are good arguments for yes and no.

The public policy objective is that when employers are responsible for the actions of employees, they will take action to reduce the risks.


The cynical view: vicarious liability exists because employers generally have deeper pockets than employees.

\end{frame}



\begin{frame}
\frametitle{\textit{John Doe v. Bennett}}

In a 2004 Supreme Court Decision, the court said:

\begin{quote}
Vicarious liability is based on the rationale that the person who puts a risky enterprise into the community may fairly be held responsible when those risks emerge and cause loss or injury to members of the public.  Effective compensation is a goal.  Deterrence is also a consideration.  The hope is that holding the employer or principal liable will encourage such persons to take steps to reduce the risk of harm in the future. Plaintiffs must show that the rationale behind the imposition of vicarious liability will be met on the facts in two respects. First, the relationship between the tortfeasor and the person against whom liability is sought must be sufficiently close. Second,  the wrongful act must be sufficiently connected to the conduct authorized by the employer. 
\end{quote}


\end{frame}



\begin{frame}
\frametitle{Vicarious Liability}

\textit{Dutton v. Bognor Regis United Building Co. Ltd.}, 1972.

This case was decided by the Court of Appeal of England.

A house was built on a rubbish deposit.

This will result in settling, but this can be dealt with through larger foundations.

By-laws required that the foundation be approved by a building inspector.

The inspector failed to make a proper inspection; part of the building collapsed.

Does vicarious liability apply here?

\end{frame}



\begin{frame}
\frametitle{Vicarious Liability}

Yes. The inspector was found responsible, but the building authority was also on the hook for the damages.

The court said:

\begin{quote}
It is at this point that I must draw a distinction between the several categories of professional men.  I can well see that in the case of a professional man who gives advice on financial or property matters -- such as a banker, a lawyer or an accountant -- his duty is only to those who rely on him and suffer financial loss in consequence.  But in the case of a professional man who gives advice on the safety of buildings, or machines, or material, his duty is to all those who may suffer injury in case his advice is bad.
\end{quote}

\end{frame}



\begin{frame}
\frametitle{Vicarious Liability}

The court also said:

\begin{quote}
If [an engineer] designs a house or a bridge so negligently that it falls down, he is liable to everyone of those who are injured in the fall:  see \textit{Clay v A.J. Crump \& Sons Ltd}.  None of those injured would have relied on the architect or the engineer.  None of them would have known whether an architect or engineer was employed or not.  But beyond doubt, the architect and engineer would be liable.  The reason is not because those injured relied on him, but because he knew, or ought to have known, that such persons might be injured if he did his work badly.
\end{quote}


\end{frame}




\begin{frame}
\frametitle{Employee Liability}


Employees, however, can also be liable in tort. 

\textit{Northwestern Mutual Insurance Co. v. J.T.O'Bryan \& Co.}, 1974

An insurance company asked the agent O'Bryan to remove an unused warehouse that had already been vandalized from a policy. 

This is a request that is standard in the industry.

The first request by letter was dated January 12th, 1970.

After two letters followed by two weeks of telephone calls, finally, an employee of the agent assured Northwestern that the risk would be removed 

It was, in fact, removed.

\end{frame}



\begin{frame}
\frametitle{\textit{Northwestern Mutual Insurance Co. v. J.T.O'Bryan \& Co.}}

Had the employee said nothing, or had the employee given a disclaimer, Northwestern would have simply cancelled the policy themselves.

Northwestern was required to pay \$22~372 when the warehouse was damaged by arson on 13 July.

Northwest claimed breach of contract with O'Bryan and negligence on the part of both O'Bryan and the employee.

\end{frame}





\begin{frame}
\frametitle{\textit{Northwestern Mutual Insurance Co. v. J.T.O'Bryan \& Co.}}

The employee was found to owe a duty of care to Northwestern for ``inexcusable negligence''.

The breach of this duty  of care was the cause of the damage; the employer was held vicariously liable.

Had the employee not been found guilty of tort, Northwestern could sue for only a breach of contract.

The contract likely had limitation clauses that would not allow Northwestern to recover all of its losses.


\end{frame}

\begin{frame}
\frametitle{Moral Hazard}

If employees know that their employers will be the ones to pay if something goes wrong, does this mean they have less incentive to avoid the problem?

(This is known as ``moral hazard''...)

\end{frame}



\begin{frame}
\frametitle{\textit{Edgeworth Construction Ltd. v. N.D. Lea \& Associates Ltd.}}

Let us consider the 1993 case of \textit{Edgeworth Construction Ltd. v. N.D. Lea \& Associates Ltd.}.

The Supreme Court of Canada ruled in this case that the engineering firm was liable in tort to the contractor because the specifications had errors~\cite{lpe}.

The case revolves around negligent misstatements, but also has a significant divergence that affects engineers.

The company was liable; but what about the engineers?

The question was, were the engineers who sealed and signed the design documents personally responsible in tort as well?

\end{frame}



\begin{frame}
\frametitle{\textit{Edgeworth Construction Ltd. v. N.D. Lea \& Associates Ltd.}}

Generally - no.

The court said:

\begin{quote}
The only basis upon which [the engineers] are sued is the fact that each of them affixed his seal to the design documents.  In my view, this is insufficient to establish a duty of care between the individual engineers and [Edgeworth Construction].  The seal attests that a qualified engineer prepared the drawing.  It is not a guarantee of accuracy.  The affixation of a seal, without more, is insufficient to found liability for negligent misrepresentation.
\end{quote} 

\end{frame}



\begin{frame}
\frametitle{\textit{Edgeworth Construction Ltd. v. N.D. Lea \& Associates Ltd.}}

This should give some reason for hope -- an engineer is not necessarily liable in tort if the outcome is bad.

So it seems that vicarious liability may not go in the opposite direction: a company that is at fault does not necessarily mean the employees are...

The distinction appears to be that there was reliance on the company rather than reliance on any individual~\cite{lpe}.

Speculate: would the ruling have been different if only one engineer had prepared and sealed the report?

\end{frame}



\begin{frame}
\frametitle{Scope of Employment}

Just because someone is employed does not mean any action the employee takes creates vicarious liability for the employer.

The test used to required a ``scope of employment'' test: the employee does what he or she is told to do.

The employer is responsible for (1) acts authorized by the employer, and (2) unauthorized acts related to the work~\cite{vicarious}.

This changed somewhat in a 1999 ruling.

\end{frame}



\begin{frame}
\frametitle{\textit{Bazley v. Curry}}

Consider the 1999 case of \textit{Bazley v. Curry}, at the Supreme Court of Canada~\cite{vicarious}.

This case focused on the subject of whether a particular act is sufficiently related to what is authorized by the employer.

If the act in question arises out of the normal duties of the job, then the employer can be vicariously liable.


\end{frame}



\begin{frame}
\frametitle{Independent Contractors}

Does liability attach where the relationship between parties is not employer-employee, but instead, one is an independent contractor?

We will return to the subject of independent contractors later...

\end{frame}


\begin{frame}
\frametitle{Vicarious Liability}

A quick summary of employer liability taken from~\cite{vl2}:

\begin{enumerate}
	\item An employer is ALWAYS DIRECTLY LIABLE for its own negligence in hiring, training, or supervising employees. 
	\item An employer is ALWAYS VICARIOUSLY LIABLE for the wrongful acts of an employee within the scope of his or her employment. 
	\item An employer MAY BE VICARIOUSLY LIABLE for the wrongful acts of an employee outside the scope of his or her employment.
	\item A party hiring an independent contractor is generally NOT VICARIOUSLY LIABLE for the wrongful acts of the independent contractor. 
	\item An employer is ALWAYS VICARIOUSLY LIABLE for wrongs committed by an agent in the scope of the agent's actual, apparent, or usual authority. 
	\item An employer MAY BE VICARIOUSLY LIABLE for an employee's breach of fiduciary duty owed to a previous employer, even if the new employer was unaware of the breach.
	\item The most effective way for employers to limit unnecessary liability is to take PROACTIVE PREVENTATIVE MEASURES. 
\end{enumerate}


\end{frame}

\begin{frame}
\frametitle{References \& Disclaimer}
\bibliographystyle{ieeetr}
\setbeamertemplate{bibliography item}{\insertbiblabel}
{\scriptsize
\bibliography{290}
}
\vfill

{\tiny Disclaimer: the material presented in these lectures slides is intended for use in the course ECE~290 at the University of Waterloo and should not be relied upon as legal advice. Any reliance on these course slides by any party for any other purpose are the responsibility of such parties.  The author(s) accept(s) no responsibility for damages, if any, suffered by any party as a result of decisions made or actions based on these course slides for any other purpose than that for which it was intended.\par}


\end{frame}


\end{document}

