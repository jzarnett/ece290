\input{configuration}

\title{Lecture 35 --- Solving Ethics Case Studies }

\author{Jeff Zarnett, based on original by Douglas Harder \\ \small \texttt{jzarnett@uwaterloo.ca} / \texttt{dwharder@uwaterloo.ca}}
\institute{Department of Electrical and Computer Engineering \\
  University of Waterloo}
\date{\today}


\begin{document}

\begin{frame}
  \titlepage

\begin{center}
  \small{Acknowledgments: Douglas Harder~\cite{dwh}, Julie Vale~\cite{jv}}
  \end{center}
\end{frame}



\begin{frame}
\frametitle{Case Studies}

The standard means by which a student is tested with respect to his or her knowledge of ethics and professional practice.

A scenario is provided and the student must identify instances of professional misconduct and must recommend ethical courses of action.

\end{frame}



\begin{frame}
\frametitle{Solution}

The solution to a case study must always be in essay form.

A correct answer that is in point form will result in a failing grade on the Professional Practice Examination.

All forms of human communication consist of:
\begin{itemize}
	\item An introduction including relevant background
	\item A body
	\item A conclusion, including a summary and recommendations
\end{itemize}

\end{frame}



\begin{frame}
\frametitle{How To Solve a Case Study}

Steps in solving a case study:

\begin{itemize}
	\item Who are the main players involved?
	\item What are their relationships?
	\item Identify relevant actions
	\item What are the relevant statutes, codes, regulations?
	\item Determine professional misconduct and ethical responses
\end{itemize}


\end{frame}



\begin{frame}
\frametitle{ACRONYMS!}

The acronym here is \textbf{PRISM}:

Steps in solving a case study:
\begin{itemize}
	\item The \textbf{P}layers involved
	\item Their \textbf{R}elationships
	\item \textbf{I}dentify their actions
	\item What \textbf{S}tatutes apply?
	\item Determine professional \textbf{M}isconduct and ethical responses
\end{itemize}

\end{frame}



\begin{frame}
\frametitle{It Could Be Worse...}

PRISM may be a forced acronym, but it's better than:
READ-EGAD!-SUMMARIZE

\begin{itemize}
	\item \textbf{R}ead
	\item \textbf{E}thical issues
	\item \textbf{G}eneration of alternatives
	\item \textbf{A}nalysis
	\item \textbf{D}ecision
	\item \textbf{S}ummary
\end{itemize}

\end{frame}



\begin{frame}
\frametitle{Case Study 1}

A case study from a previous Professional Practice Examination:


	McGee is a professional engineer who is employed on a full-time basis by  EngrCIS, Inc., a large engineering firm.  However, for a number of reasons, McGee is unhappy and for some time has been thinking about looking for a new job.  Although McGee's current employment at EngrCIS provides good pay and interesting work, McGee is finding it difficult to work with DiNozzo, a professional engineer who is McGee's supervisor at EngrCIS.
\end{frame}



\begin{frame}
\frametitle{Case Study 1}

A case study from a previous Professional Practice Examination:


In the year since McGee joined the company, DiNozzo has frequently made derogatory jokes and remarks about McGee's race and religion -- sometimes even in meetings with other engineers and clients. On many occasions, McGee has informed DiNozzo that such remarks are offensive, hurtful, and inappropriate and has asked DiNozzo to stop.  DiNozzo refuses to do so and says that McGee should ``toughen up and learn to take a joke'' if McGee expects to have a successful career at EngrCIS. 
\end{frame}



\begin{frame}
\frametitle{Case Study 1}


Recently, McGee met with a professional engineer colleague, Sciuto, who is a vice president at NYCE, Ltd., another engineering company.  Upon hearing that McGee was interested in considering other opportunities, Sciuto offered McGee a part-time job to work in the evenings and on weekends on a trial basis as an engineer for NYCE.  McGee would work under Sciuto's supervision with the intent that in a few months, if McGee preferred working at NYCE, McGee would resign from EngrCIS and become a full-time employee of NYCE.



\end{frame}



\begin{frame}
\frametitle{Start with P}

Remember the P from PRISM: Who are the Players?


\end{frame}


\begin{frame}
\frametitle{Start with P}

Remember the P from PRISM: Who are the Players?

\begin{itemize}
	\item McGee, P.Eng.
	\item DiNozzo, P.Eng.
	\item EngrCIS
	\item Sciuto, P.Eng.
	\item NYCE
\end{itemize}


\end{frame}



\begin{frame}
\frametitle{Now R}

Next is R:  What are the relationships between the parties? 


\end{frame}

\begin{frame}
\frametitle{Now R}

Next is R:  What are the relationships between the parties? 

\begin{itemize}
	\item McGee  and DiNozzo are employed by EngrCIS
	\item DiNozzo is McGee's supervisor 
	\item McGee is colleagues with Sciuto
	\item The colleague is a VP at NYCE
	\item McGee has the option of contracting with NYCE
\end{itemize}


\end{frame}

\begin{frame}
\frametitle{Next I}

Next I: What are the actions?


\end{frame}

\begin{frame}
\frametitle{Next I}

Next I: What are the actions?

\begin{itemize}
	\item DiNozzo has been harassing McGee
	\item The VP has made a contract offer to McGee
\end{itemize}

\end{frame}



\begin{frame}
\frametitle{Then S}

What are the relevant statues, codes, regulations?

\end{frame}



\begin{frame}
\frametitle{Then S}

In this case, the relevant statutes, codes, regulations, etc., include:
\begin{itemize}
	\item The Professional Engineers Act, including
	\begin{itemize}
		\item The requirement that a Certificate of Authorization is required for any engineering services offered to the public
		\item The Code of Ethics
		\item The definition of professional misconduct
	\end{itemize}
	\item The Ontario Human Rights Code
	\item The Ontario Occupational Health and Safety Act
\end{itemize}

\end{frame}



\begin{frame}
\frametitle{Statutes, Codes, Regulations...}

To the practitioner's employer, he or she:

Has a duty to act at all times with fairness and loyalty (77.1);

Shall act in professional engineering matters:\\
\quad As a faithful agent or trustee, and\\
\quad Shall regard as confidential information obtained by the practitioner as to the business affairs, technical methods or processes of an employer (77.3); and


Shall avoid or disclose a conflict of interest that might influence the practitioner's actions or judgment (77.3)


\end{frame}



\begin{frame}
\frametitle{Statutes, Codes, Regulations...}

To other members of the profession, he or she has a duty to act at all times with fairness, loyalty (77.1), courtesy and good faith (77.7);


Shall not:
\begin{itemize}
\item Accept an engagement to review the work of another for the same employer except with the knowledge of the other practitioner or except where the connection of the other practitioner with the work has been terminated, nor
\item Maliciously injure the reputation of another practitioner (77.7); and
\end{itemize}
Shall:
\begin{itemize}
\item Give proper credit for engineering work,
\item Provide opportunity for professional development and advancement of the practitioner's associates and subordinates, and
\item Extend the effectiveness of the profession through the interchange of engineering information and experience (77.7). 
\end{itemize}


\end{frame}



\begin{frame}
\frametitle{Statutes, Codes, Regulations...}

Recall the Certificate of authorization rules:

12. (2) No person shall offer to the public or engage in the business
of providing to the public services that are within the practice of
professional engineering except under and in accordance with a certificate of authorization. 

\end{frame}



\begin{frame}
\frametitle{Statutes, Codes, Regulations...}

The Human Rights Code prohibits discrimination
on the the following grounds:

\begin{tabular}{l l l}
		Race & Ancestry	& Place of origin \\
		Colour & Ethnic origin & Citizenship\\
		Creed & Sex & Sexual orientation\\
		Gender identity & Gender expression	& Age\\
		Marital status & Family status & Disability\\
		Record of Offences & Reprisal & Association\\
		The receipt of public assistance & &\\
\end{tabular}

Oxford English Dictionary: Discrimination, \textit{n}.\\
	The action of discriminating; the perceiving, noting, or making a distinction
	or difference between things; a distinction (made with the mind, or in action). 

\end{frame}



\begin{frame}
\frametitle{Statutes, Codes, Regulations...}

Occupational Health and Safety Act:


\textbf{Policies, violence and harassment}\\
	32.0.1(1)(b) An employer shall prepare a policy with respect to
harassment.

\textbf{Program, harassment}\\
	32.0.6  (1)  An employer shall develop and maintain a program to implement the policy with respect to workplace harassment required under clause 32.0.1(1)(b).

\textbf{Contents}\\
	(2)  Without limiting the generality of subsection (1), the program shall,
\begin{enumerate}[(a)]
\item include measures and procedures for workers to report incidents of workplace harassment to the employer or supervisor;
\item set out how the employer will investigate and deal with incidents and complaints of workplace harassment; and
\item include any prescribed elements.
\end{enumerate}

\end{frame}



\begin{frame}
\frametitle{Finally M}

M is for Misconduct; what do you find in the case study?

\end{frame}




\begin{frame}
\frametitle{Finally M}

One instance of professional misconduct is DiNozzo's harassment of McGee.

\begin{itemize}
\item Such harassment constitutes a violation of the
Ontario Human Rights Code as well as professional
misconduct
\item For whatever reason, it seems that McGee
has not approached H.R. at EngrCIS
\item Given he appears to enjoy working at EngrCIS,
it would be reasonable to give them the
opportunity to correct this violation
\item If H.R. does not respond, McGee should
file a complaint with PEO
\item Depending on the severity, a last step might
be to file a human rights complaint
\end{itemize}

\end{frame}



\begin{frame}
\frametitle{More Ms}

McGee is being offered a contract by NYCE.
\begin{itemize}
\item The scenario suggests that he would be
providing engineering services to NYCE
\item Providing such services requires a Certificate of Authorization
\item Providing such services without a C. of A.
would be a breach of Section 12(2) the Act
\item A breach of the Act is professional misconduct
under 72(2)(g)
\item McGee should acquire a C. of A. if he intends
to enter into such a contract
\end{itemize}

\end{frame}



\begin{frame}
\frametitle{M also for Moonlighting?}

The scenario suggests that he would be providing engineering services to NYCE.

\begin{itemize}
\item The Code of Ethics requires McGee to
Provide, in writing, a statement to NYCE that he is
employed by EngrCIS (even if this is obvious)
\item Ensure that there is no conflict in the work 
performed
\item Inform EngrCIS that he is entering into a contract
with NYCE
\item Failure to do so would be professional
misconduct under 72(2)(i) as an
undisclosed conflict of interest
\end{itemize}

\end{frame}




\begin{frame}
\frametitle{Writing It Out}

If you are simply required to respond to a case study, the response would be an essay-based which would cover the points we have just discussed.

Every essay must have an introduction, a body, and a conclusion.

The introduction would briefly cover the situation and explain the approach that the body of the essay will take.

In each case, we will look at the obligations under the Code of Ethics and issues with respect to any statutes that may have been breached.

We conclude by summarizing the actions that should be taken.

\end{frame}



\begin{frame}
\frametitle{Case Study 1}

In the professional practice examination, there were three questions asked:

\begin{enumerate}[(a)]
\item Comment on DiNozzo's conduct with respect to Ontario Regulation 941.
\item In relation to the regulation of the practice of professional engineering what should McGee consider doing about DiNozzo's conduct?
\item Specify and explain the requirements, if any, that McGee must satisfy in order to properly undertake such part-time employment with NYCE.
\end{enumerate}

\end{frame}



\begin{frame}
\frametitle{Notes on Case Study 1}

In these questions, it specifically restricts the response with respect to DiNozzo's actions to the regulations.

In this case, it would be therefore only a question of professional misconduct by harassment.

 (The OHSA and Ontario Human Rights Code are not usually tested in the PPE).

\end{frame}



\begin{frame}
\frametitle{Case Study 2}

Another Case Study from a previous PPE:


Gibbs, the owner of a house in the City of Waterloo, was notified by the city that the condition of the foundation walls of his house violated the standards set out in the city's property standards by-law.  The city, being concerned that the foundation walls had deteriorated to the point of being structurally unsafe, ordered Gibbs to obtain a written report by a professional engineer as to the condition of the walls.  Fornell prepared a report stating that he had inspected the foundation and that the foundation walls appeared to be ``structurally sound and capable of safely sustaining the house for many more years.''


\end{frame}



\begin{frame}
\frametitle{Case Study 2}

Gibbs submitted Fornell's report to the city.  In response, the city sent a letter to Gibbs with a copy to Fornell pointing out the city's observations regarding the deterioration of the walls, including evidence of significant water permeation, together with photographs taken by the city's inspector. In the letter, the city requested the condition of the foundation be reassessed and a response be made to the city within two weeks. Gibbs was unaware that Fornell would be waiting for authorization for him to spend more time on the project and accordingly did not contact Fornell and request him to respond. Fornell did not follow up with either Gibbs or the city.

\end{frame}



\begin{frame}
\frametitle{Case Study 2}

Following a second request to Gibbs, copied to Fornell, Fornell responded by letter to the city, advising that he had never examined the interior of the walls, only the exterior and admitted the photographs provided by the city indicated that the foundation was structurally unsound.


\end{frame}



\begin{frame}
\frametitle{Case Study 2 Questions}

In the professional practice examination, there were two questions asked: 

\begin{enumerate}[(a)]
\item Comment on the services provided by Fornell, in relation to Regulation 941. In your answer, also discuss Fornell's conduct regarding his dealings with the City.
\item Fornell does not have a certificate of authorization.  Does Fornell need one under the facts described above?  Explain why or why not.  What are the possible consequences to a professional engineer of acting without a certificate of authorization when one is required?
\end{enumerate}

\end{frame}



\begin{frame}
\frametitle{Case Study 3}

A third case study from a previous Professional Practice Examination:

	Duckie, P.Eng., a senior professional engineer, established a small firm, Mallard Engineering, to provide professional engineering services to the public.  The firm became busy very quickly and within a few months, he hired Palmer1, a bright recent university graduate with an engineering degree, to assist with the work.  Duckie strongly believed in mentoring and hoped that in several years, after obtaining the necessary experience requirements and becoming a professional engineer, Palmer would assume increasing managerial responsibility and possibly an ownership interest in the firm.

\end{frame}



\begin{frame}
\frametitle{Case Study 3}

After about a year after Palmer joined the firm, Mallard Engineering was asked by one of its clients to provide a formal report that included an engineering option.  Palmer performed the work on that matter and prepared a draft of the report.  Before having a chance to review Palmer's work, Duckie received an urgent request from another client that required Duckie to leave on a lengthy business trip.  On the way out of the office, Duckie stopped at Palmer's desk and said, ``Sorry, but I'll be out of the country and tied up completely for the next three weeks, so I won't be able to review that report.  I know that it's due tomorrow, so go ahead and sign it under your own name and send it to the client so that we can meet the deadline.''  Duckie was confident that that would be all right, since Palmer has always produced outstanding work in the past.  Palmer proceed to complete the report, signed it ``J. Palmer, Eng., Mallard Engineering'' and sent it to the client.


\end{frame}



\begin{frame}
\frametitle{Case Study 3}

In the professional practice examination, there were three questions asked: 
\begin{enumerate}[(a)]
\item Discuss the conduct of both Duckie and Palmer.  What, if anything, should they be concerned about?
\item Could Duckie or Palmer be subject to a hearing by the Discipline Committee of PEO?  Discuss.
\item Is there anything about Duckie's conduct relative to the Code of Ethics that is commendable?
\end{enumerate}

\end{frame}



\begin{frame}
\frametitle{Case Study 4}

One final example from a previous PPE:

IBF Engineering, a large engineering firm, was hired to prepare the design for a chemical production plant for AIC, Inc.  In addition to preparing the plant design, IBF Engineering's duties included providing inspection services during the construction stage of the project.  The project was completed successfully.
I

\end{frame}



\begin{frame}
\frametitle{Case Study 4}

You are a professional engineer and have been employed on a full-time basis for IBF Engineering for several years.  You work in the Process Division and are involved on several process design projects.  You were an important member of the design team that prepared the design for AIC Inc.'s plant.  In addition to working for IBF Engineering, you supplement your income by occasionally undertaking work on weekends and during evenings for Marine Engineering, Ltd., another engineering company.  A colleague of yours, who is a professional engineer at Marine Engineering, Ltd., assigns you such work and assumes responsibility for it.


\end{frame}

\begin{frame}
\frametitle{Case Study 4}

	A few years after the plant was completed, AIC, Inc., decides to restructure its operations and sell the plant.  ASN Corp. has agreed to buy the plant, but before it does so, ASN Corp. wants to satisfy itself (and its bank) that the plant was built to proper standards and is in good physical condition.  ASN Corp. hires Marine Engineering, Ltd. to inspect the physcial plant and to review relevant documents (including the original plans, ``as-built'' drawings, and operations and maintenance logs). Marine Engineering, Ltd. is very busy on several projects and asks you to assist with the plant inspection and document review.

\end{frame}




\begin{frame}
\frametitle{Case Study 4}


In the professional practice examination, there were three questions asked:

\begin{enumerate}[(a)] 
\item Discuss the appropriateness of your employment arrangements.
\item Assuming that your employment arrangements have not changed since the plant was designed and constructed, discuss how you respond to Marine Engineering, Ltd.'s request for assistance.
\item Would you need a Certificate of Authorization to provide services to Marine Engineering, Ltd.?  Explain.
\end{enumerate}

\end{frame}


\begin{frame}
\frametitle{References \& Disclaimer}
\bibliographystyle{ieeetr}
\setbeamertemplate{bibliography item}{\insertbiblabel}
{\scriptsize
\bibliography{290}
}
\vfill

{\tiny Disclaimer: the material presented in these lectures slides is intended for use in the course ECE~290 at the University of Waterloo and should not be relied upon as legal advice. Any reliance on these course slides by any party for any other purpose are the responsibility of such parties.  The author(s) accept(s) no responsibility for damages, if any, suffered by any party as a result of decisions made or actions based on these course slides for any other purpose than that for which it was intended.\par}


\end{frame}


\end{document}

