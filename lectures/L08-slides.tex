\input{configuration}

\title{Lecture 8 --- Contracts: Recission }

\author{Jeff Zarnett \\ \small \texttt{jzarnett@uwaterloo.ca}}
\institute{Department of Electrical and Computer Engineering \\
  University of Waterloo}
\date{\today}


\begin{document}

\begin{frame}
  \titlepage

\begin{center}
  \small{Acknowledgments: Douglas Harder~\cite{dwh}, Julie Vale~\cite{jv}}
  \end{center}
\end{frame}



\begin{frame}
\frametitle{Recission}

When parties enter into a contract, there may be circumstances where on the basis of equity, the court may allow \alert{recission}.

This is when one party can rescind the contract.\\
\quad A remedy for harm caused to one party of the contract.

This can be done at any time and it is an \alert{unwinding} of the contract.\\
\quad The goal is to return all parties to their original state before the contract.

This can also be referred to as cancellation, reversing, etc.

\end{frame}



\begin{frame}
\frametitle{Cause for Recission}

There must be an appropriate basis for the court to make this determination.

As before, the court will not bail someone out of a ``bad deal''.

The grounds for recission we will examine are:

\begin{enumerate}
	\item Misrepresentation
	\item Duress
	\item Economic duress
	\item Undue influence
\end{enumerate}

\end{frame}



\begin{frame}
\frametitle{Misrepresentation}

Misrepresentation is a false assertion of fact which induces another party to enter into a contract~\cite{lba}.

If it is made with knowledge that it is false, without an honest belief in its truth, or recklessly, then it is a \textit{fraudulent} misrepresentation.

Otherwise it is an \alert{innocent} misrepresentation. 

It is the duty of one who made an innocent misrepresentation and learns of its falsity, to inform he other party of the true situation.

Failure to do so makes the innocent misrepresentation into a fraudulent one.

\end{frame}



\begin{frame}
\frametitle{Misrepresentation}

If a party is induced by a misrepresentation to enter into a contract, he or she is the \alert{deceived party}.

The deceived party may apply for recission of the contract. 

If the court finds in that party's favour, the court will give the deceived party the option of voiding the contract.

The deceived must repudiate the contract promptly; excessive delay or taking further benefits of the contract removes this choice.

\end{frame}



\begin{frame}
\frametitle{Damages}

Other remedies may be available, such as compensation, for any losses that have resulted as a consequence of the contract.

If it is fraudulent misrepresentation, the deceived party may sue for additional damages under tort law (we'll come back to that...).

Fraud is also a criminal act.\\
\quad And can lead to jail time!

\end{frame}



\begin{frame}
\frametitle{\textit{Township of McKillop v Pidgeon and Foley}, 1908}

Engineering contract example related to excavation.

The specification gave the estimated amount of excavation required for a particular project.

The contractor submitted a tender based on that submission.

It was found that the estimation was 16\% less than that required.

This would have eaten up any profit the contractor would have received.

The contractor sought to rescind the contract. What did the court say?

\end{frame}



\begin{frame}
\frametitle{\textit{Township of McKillop v Pidgeon and Foley}, 1908}

The court found this to be an innocent misrepresentation. 

The contractor was entitled to recission; they could repudiate the contract.


\end{frame}



\begin{frame}
\frametitle{Duress}

The TV show standard of duress: someone pointing a gun at someone else.

The actual definition is actual or threatened violence or imprisonment as a means of forcing someone to enter into a contract.

The threat of violence may not be against the party directly; it could be against a family member, for example.

\end{frame}



\begin{frame}
\frametitle{Duress}

\textit{Mutual Finance Co. Ltd. v. John Wetton \& Sons Ltd.}

One party coerced the other into a second guarantee by threatening to reveal that the threatened party had forged the first guarantee.

Important take-away: threatening to reveal a crime is a form of duress.

The contract here was ruled unenforceable due to duress.

\end{frame}



\begin{frame}
\frametitle{Economic Duress}

The courts now recognize the concept of economic duress, as similar to, but distinct from the regular form.

Duress is a threat of violence or imprisonment; economic duress is about financial losses rather than physical ones.

The Canadian Supreme Court case that set the precedent in 1993 was \textit{Gotaverken Energy Systems Ltd. v Cariboo Pulp \& Paper Co.}

\end{frame}



\begin{frame}
\frametitle{Economic Duress}

The owner required a refit of a boiler.

The contractor would be paid in excess of \$24 million.

The project had to be completed during a shutdown period and the contractor was to work two 11-hour shifts seven days a week.

Problems that were the responsibility of the owner resulted in substantial losses for the contractor during the project.

The owner was ready to acknowledge these (and pay).

The contractor, however, threatened to reduce the working hours from 154~h/wk to 36.5~h/wk unless the contract was rewritten to pay an additional \$10 million.

\end{frame}



\begin{frame}
\frametitle{Economic Duress}

The court ruled that this was economic duress.

The owner was threatened with substantial business losses if the contractor worked on the new, lighter schedule.

The contractor was to be compensated with an additional \$6 million, not the \$10 million they had wanted.


\end{frame}



\begin{frame}
\frametitle{Undue Influence}

When one party has significant influence on another in forcing the other party to enter into a contract, it is said to be undue influence.

Undue influence arises where there is a relationship between the parties where one has a special skill or knowledge causing the other to trust him/her~\cite{lba}.

Examples of this relationship include:
\begin{itemize}
	\item Doctor and patient
	\item Lawyer and client
	\item Parent and child
\end{itemize}

\end{frame}



\begin{frame}
\frametitle{Undue Influence}

Undue influence can also arise when one party is in dire straits. Example~\cite{lba}:

A ship is at sea and runs aground. The coast is barren and ``Winter is Coming''.

A rescue crew comes along and offers to help, but they want to buy the valuable cargo of the wrecked ship at a bargain price (to which the captain agrees).

The owners of the wrecked ship then repudiate the contract.

\end{frame}



\begin{frame}
\frametitle{Undue Influence}

This is undue influence: the wrecked ship's crew is in serious trouble and they are in need of a rescue.

The rescuing ship is in a much better position and can prey upon the desperation of the shipwrecked sailors.

Interesting question: why is this undue influence and not duress?

\end{frame}



\begin{frame}
\frametitle{Take Precautions}

This problem can be avoided by taking adequate precautions.

Ensuring the other party seeks independent legal advice on the contract will usually refute any allegation of undue influence.

This may be expensive, but it is dramatically cheaper than litigation!

(Seeking independent legal advice is almost always a good idea, actually...)


\end{frame}


\begin{frame}
\frametitle{References \& Disclaimer}
\bibliographystyle{ieeetr}
\setbeamertemplate{bibliography item}{\insertbiblabel}
{\scriptsize
\bibliography{290}
}
\vfill

{\tiny Disclaimer: the material presented in these lectures slides is intended for use in the course ECE~290 at the University of Waterloo and should not be relied upon as legal advice. Any reliance on these course slides by any party for any other purpose are the responsibility of such parties.  The author(s) accept(s) no responsibility for damages, if any, suffered by any party as a result of decisions made or actions based on these course slides for any other purpose than that for which it was intended.\par}


\end{frame}


\end{document}

