\input{configuration}

\title{Lecture 15 --- Tort: Duty to Warn - Economic Loss; Occupier's Liability }

\author{Jeff Zarnett \\ \small \texttt{jzarnett@uwaterloo.ca}}
\institute{Department of Electrical and Computer Engineering \\
  University of Waterloo}
\date{\today}


\begin{document}

\begin{frame}
  \titlepage

\begin{center}
  \small{Acknowledgments: Douglas Harder~\cite{dwh}, Julie Vale~\cite{jv}}
  \end{center}
\end{frame}

\part{Duty to Warn - Economic Loss}

\begin{frame}
\partpage
\end{frame}


\begin{frame}
\frametitle{Duty to Warn}

The previous discussion centred around a duty to warn that a particular product might cause injury.

Product liability was not extended to economic losses, in the absence of physical injury, until 1973.

The case: \textit{Rivtow Marine Ltd. v. Washington Iron Works et al.}

The ruling: economic losses caused by the use of a defective product may, in some circumstances, be recoverable~\cite{lpe}. 

\end{frame}



\begin{frame}
\frametitle{\textit{Rivtow Marine Ltd. v. Washington Iron Works et al.}}

Description from~\cite{lpe}.

The plaintiff hired a logging barge with a crane from Washington Iron Works.

Washington Iron Works had also manufactured a second crane, virtually identical to the first, on a similar barge.

The second crane collapsed, and an operator was killed.

Very serious structural defects were found to be responsible for the collapse.

\end{frame}



\begin{frame}
\frametitle{\textit{Rivtow Marine Ltd. v. Washington Iron Works et al.}}

Similar structural defects were found in the crane the plaintiff hired.

The defendants had been aware that the cranes were subject to cracking due to negligence in the design. 

Neither defendant took steps to warn the plaintiff of the potential danger and the need for repair.

\end{frame}



\begin{frame}
\frametitle{\textit{Rivtow Marine Ltd. v. Washington Iron Works et al.}}

The Supreme Court held that the manufacturers were liable in tort.

They had a duty to warn the plaintiff of the danger.

And having failed to do so, they owed the plaintiffs for the lost profits while the crane was undergoing repairs.

\end{frame}



\begin{frame}
\frametitle{\textit{MacMillan Bloedel Ltd. v. Foundation Co.}}

A 1977 case: \textit{MacMillan Bloedel Ltd. v. Foundation Co.}~\cite{lpe}.

The defendant's workers negligently damaged an electric cable supplying power to the office building of the plaintiff.

The plaintiff could not continue work and sent its employees home.

The plaintiff sued for the salaries and wages paid to its staff (\$48~841.00).

Did this action succeed?

\end{frame}



\begin{frame}
\frametitle{\textit{MacMillan Bloedel Ltd. v. Foundation Co.}}

Let's think about some of the important issues.

Question 1: Did the plaintiff suffer a loss?

Question 2: Was the loss a result of the defendant's actions?

Question 3: Was this foreseeable on the part of the defendant?

\end{frame}



\begin{frame}
\frametitle{\textit{MacMillan Bloedel Ltd. v. Foundation Co.}}

The court agreed that the defendant's actions were negligent.

The court also found this was foreseeable.

But -- the court was not satisfied that the loss was the direct result of the defendant's actions.

The plaintiff was going to pay the employee salaries anyway, and did not present additional evidence of further economic losses. 

\end{frame}



\begin{frame}
\frametitle{\textit{MacMillan Bloedel Ltd. v. Foundation Co.}}

Remember that causation is an important part of tort law. 

Injuries, or damages must be ``directly caused'' by the defendant's actions, inactions, or negligence. 

The salaries were going to be paid anyway, so the court's decision turned on the fact that the plaintiff could not demonstrate it was harmed.

If there were damages directly caused, they would be recoverable.

\end{frame}



\begin{frame}
\frametitle{\textit{Bethlehem Steel Corp. v. St. Lawrence Seaway Authority}}

There was, however, another case that limited the concept: \\
\quad \textit{Bethlehem Steel Corp. v. St. Lawrence Seaway Authority}~\cite{lpe}.

A ship ran into a bridge over a canal, destroying the bridge, and obstructing the canal.

Shipping was delayed for several days.

There were two claims that were interesting:

\begin{enumerate}
	\item Lost profits for two ships delayed about two weeks.
	\item Costs of shipping cargo over land instead.
\end{enumerate}

Which of these succeeded?

\end{frame}



\begin{frame}
\frametitle{\textit{Bethlehem Steel Corp. v. St. Lawrence Seaway Authority}}

Neither one!

The court held that the \textit{Rivtow Marine} case did not change the law.

Instead, it extended liability to economic loss where there was physical harm or the threat thereof to property.

There was no threat of harm here; only delay.

\end{frame}



\begin{frame}
\frametitle{\textit{Junior Books Ltd. v. Veitchi Co. Ltd}}

A 1982 decision from the English House of Lords~\cite{lpe}. 

A negligently laid floor was defective but was not dangerous, nor did it risk damages to property.

The floor needed to be replaced, materials moved out of the space, profits lost due to business disturbance, et cetera.

The damage caused to the owner was a direct result of the negligence.

The defendant was, in fact, found liable.


\end{frame}



\begin{frame}
\frametitle{\textit{Canadian National Railway v. Norsk Pacific Steamship Co.}}

This 1990 case referred to the \textit{Junior Brooks} decision~\cite{lpe}.

A barge collided with a bridge owned by Public Works Canada, which CNR used to cross the Fraser River.

CNR did not own the bridge, but they owned the tracks to/from the bridge. 

The court believed there was enough ``proximity'' between CNR and the defendants to justify liability.

CNR was so closely linked with Public Works Canada that it was possible to demonstrate the harm was foreseeable on the part of the captain of the barge.

Given this, CNR was entitled to recover its economic losses.


\end{frame}


\part{Occupier's Liability}

\begin{frame}
\partpage
\end{frame}



\begin{frame}
\frametitle{Occupier's Liability}

Common sense might lead us to think that the occupier (owner or tenant) of land/buildings would be liable for injuries according to the rules of negligence.

Nope! Occupier's Liability comes from Land Law, which finds its origin in the old common law principles, which is described in~\cite{lba} as ``rigid and irrational''.

The occupier must ensure the safety of individuals coming into the property. 

What about trespassers?

\end{frame}



\begin{frame}
\frametitle{Occupier's Liability}

Obviously, the highest standard of care applies to those who are there legitimately, whether for business purposes or as guests.

Still, a duty of care extends to trespassers.\\
\quad One cannot deliberately harm trespassers (e.g., set traps)~\cite{lpe}. 

The duty is owed by the occupier to deal with hazards of which he/she is aware, or ought reasonably to be aware of.

The relevant statute is the Occupier's Liability Act in Ontario.

\end{frame}



\begin{frame}
\frametitle{Ice, Ice Baby}

For example: ice buildup on paths or sidewalks must be cleared or salted.

What if someone slips and falls, and suffers an injury?

Also suitable: warning signs (think: duty to warn).

The occupier may escape liability if the work has been contracted out to an independent contractor and ensured that the contractor is competent.


\end{frame}


\begin{frame}
\frametitle{Some Excerpts from the Act}

Occupier's duty (Section 3.(1): 
\begin{quote}
An occupier of premises owes a duty to take such care as in all the circumstances of the case is reasonable to see that persons entering on the premises, and the property brought on the premises by those persons are reasonably safe while on the premises.
\end{quote}


Assumption of risk:
\begin{quote}
The duty of care provided for in subsection 3 (1) does not apply in respect of risks willingly assumed by the person who enters on the premises, but in that case the occupier owes a duty to the person to not create a danger with the deliberate intent of doing harm or damage to the person or his or her property and to not act with reckless disregard of the presence of the person or his or her property.
\end{quote}


\end{frame}


\begin{frame}
\frametitle{Occupier's Liability}

For discussion: an employee reports that the ventilation system at the office is expelling black dust, believed to be causing respiratory illness.

Assume the building is rented by the company.

Does the company have a responsibility to do something about this?

Does the landlord?


\end{frame}

\begin{frame}
\frametitle{References \& Disclaimer}
\bibliographystyle{ieeetr}
\setbeamertemplate{bibliography item}{\insertbiblabel}
{\scriptsize
\bibliography{290}
}
\vfill

{\tiny Disclaimer: the material presented in these lectures slides is intended for use in the course ECE~290 at the University of Waterloo and should not be relied upon as legal advice. Any reliance on these course slides by any party for any other purpose are the responsibility of such parties.  The author(s) accept(s) no responsibility for damages, if any, suffered by any party as a result of decisions made or actions based on these course slides for any other purpose than that for which it was intended.\par}


\end{frame}


\end{document}

