\input{configuration}

\title{Lecture 7 --- Contracts: Mistake }

\author{Jeff Zarnett \\ \small \texttt{jzarnett@uwaterloo.ca}}
\institute{Department of Electrical and Computer Engineering \\
  University of Waterloo}
\date{\today}


\begin{document}

\begin{frame}
  \titlepage

\begin{center}
  \small{Acknowledgments: Douglas Harder~\cite{dwh}, Julie Vale~\cite{jv}}
  \end{center}
\end{frame}


\begin{frame}
\frametitle{Only Human}

As we know from the parol evidence rule, communications prior to the written contract cannot generally affect the terms.

We also examined a few exceptions. 

People, however, make mistakes. What happens if there is a mistake in a written contract?

\end{frame}


\begin{frame}
\frametitle{``That was easy!''}

If a mistake is discovered in the contract and both parties agree to amend the contract to fix this error, that's all it takes.

That may be marking up and initialing the change on the original document or a new copy of the contract with the mistake fixed.

That's the easy case: everyone gets along. 

\end{frame}

\begin{frame}
\frametitle{Common Clerical Mistakes}

When both parties have come to an agreement but the mistake is only introduced when the contract is written, this is a \alert{common mistake}.

It was a mistake by both parties in the preparation of the agreement.

A clerical error will allow a party to apply to a court to \alert{rectify} the contract.

\end{frame}


\begin{frame}
\frametitle{Rectification}

\begin{center}
	\includegraphics[width=0.8\textwidth]{images/bridgerectifier.png}\\
	Bridge Rectifier~\cite{rectifier}
\end{center}

\end{frame}

\begin{frame}
\frametitle{References \& Disclaimer}
\bibliographystyle{ieeetr}
\setbeamertemplate{bibliography item}{\insertbiblabel}
{\scriptsize
\bibliography{290}
}
\vfill

{\tiny Disclaimer: the material presented in these lectures slides is intended for use in the course ECE~290 at the University of Waterloo and should not be relied upon as legal advice. Any reliance on these course slides by any party for any other purpose are the responsibility of such parties.  The author(s) accept(s) no responsibility for damages, if any, suffered by any party as a result of decisions made or actions based on these course slides for any other purpose than that for which it was intended.\par}


\end{frame}


\end{document}

